\section{Інформаційне забезпечення}
\subsection{Вхідні дані}


Дані тренувань:
\begin{itemize}
  \item Загальна кількість випробувань;
  \item Кількість правильно виявлених ознак;
  \item Кількість неправильно виявлених (за виключенням пропущених) ознак;
  \item Кількість пропущених ознак;
\end{itemize}

\subsection{Вихідні дані}

Вихідними є всі дані тренувань, та висновки з них щодо успішності тренувань, що можуть буди цікавими користувачу. Перегляд вихідних даних виконується через інтерфейс історії тренувань.
Висновкові дані:
\begin{itemize}
  \item Процент успішності сеансу тренування
  \item Прогрес успішності в заданому режимі за проміжок часу
  \item Максимальні показники успішності в заданому режимі
\end{itemize}

\subsection{Опис структури бази даних}

У системі використовується нереляційна база даних типу key-value store такої структури:
\begin{itemize}
  \item Список ключів тренувальних сесій
  \item Співставлення кожному ключу сесії даних тренування
  \item Співставлення кожному ключу сесії її дати
  \item Співставлення кожному ключу сесії її конфігурації
  \item Співставлення кожній конфігурації множини ключів сесій
\end{itemize}

\begin{center}
  \includegraphics[scale=0.6]{./db.png}
\end{center}
