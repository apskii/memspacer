\section{Технологічний розділ}
\subsection{Керівництво користувача}
Є два способи роботи із системою:
\begin{itemize}
  \item Через інтерфейс користувача.
  \item Безпосередньо через виконуваний файл тренажеру.
\end{itemize}

Перший спосіб є стандартним і рекомендованим усім звичайним користувачам, другий є досить низкорівневим, і призначеним в першу чергу для розробників альтернативних користувацьких інтерфейсів для тренажеру.
\subsubsection{Робота із системою через інтерфейс користувача}
Робота із системою цим шляхом передбачає запуск виконуваного файлу ``memspacer-ui'' (або ``memspacer-ui.exe'' на системах \emph{Windows}). Після цього користувач потрапляє до інтерфейсу, наведеного на рисунку ~\ref{fig:memspacer-ui}.
\begin{figure}[here]
  \caption{Інтерфейс користувача}
  \centering\includegraphics[scale=0.6]{./memspacer-ui.png}
  \label{fig:memspacer-ui}
\end{figure}

Вікно інтерфейсу має дві вкладки, назви яких відповідають їх призначенню:
\begin{itemize}
  \item Налаштування.
  \item Істория тренувань.
\end{itemize}

Випадаючий список над вкладками призначений для вибору профілю, а кнопки після нього, зліва-направо, для досягнення таких цілей:
\begin{itemize}
  \item Створення нового профілю.
  \item Збереження конфігурації поточного профілю.
  \item Відміна змін до конфігурації поточного профілю.
\end{itemize}
\newpage
Призначення інших елементів інтерфейсу наведено у таблиці:
\small\begin{longtable}{| C{6cm} | C{3cm} | C{6cm} |}
  \hline
  Елемент & Тип & Призначення \\
  \hline
  Режим :: Звичайний K-N-Back
  & Альтернатива (``режим'')
  & Використовувати під час тренування звичайний K-N-Back на одній грані \\
  \hline
  Режим :: K-N-Back з обертами і здвигами за модулем
  & Альтернатива (``режим'')
  & Використовувати під час тренування просторовий K-N-Back
  з обертами і здвигами за модулем \\
  \hline
  Довжина буфера :: N
  & Число
  & Задати кількість необхідних для запамятовування
  елементів буферу кожної із ознак \\
  \hline
  Додаткові просторові перетворення :: Z-оберти лицевою гранню
  & Флаг
  & Використовувати під час тренування повороти кубу навколо осі,
  перпендикулярної площині екрану \\
  \hline
  Додаткові ознаки :: Колір
  & Флаг
  & Використовувати під час тренування блимання клітин різними кольорами
  у якості додактової ознаки до запам'ятання \\
  \hline
  Додаткові ознаки :: Звук
  & Флаг
  & Використовувати під час тренування оголошення літер
  у якості додактової ознаки до запам'ятання \\
  \hline
  Графічні опції :: Космічний фон
  & Флаг
  & Використовувати під час тренування шейдер космосу для фону \\
  \hline
  Графічні опції :: Колір спокою
  & Колір
  & Колір, у якому перебуває клітина у стані спокою (не блимає)\\
  \hline
  Графічні опції :: Колір активації
  & Колір
  & Колір, якого набуває клітина під час блимання \\
  \hline
  Таймінги :: Час активації
  & Число (секунди)
  & Час, за який клітина набуває кольору блимання під час блимання \\
  \hline
  Таймінги :: Інтервали
  & Число (секунди)
  & Інтервали між демонстраціями рядів ознак \\
  \hline
\end{longtable}\normalsize
\subsubsection{Робота із системою безпосередньо через винонуваний файл програми моделювання}
Робота із системою цим шляхом передбачає запуск виконуваного файлу ``memspacer'' (або ``memspacer.exe'' на системах \emph{Windows}). Конфігурування здійснюється через аргументи командної строки, результати тренування виводяться на стандартний пристрій виводу після закінчення тренування.
\newpage
Опис параметрів командної строки наведено у таблиці:
\small\begin{longtable}{| C{4cm} | C{5cm} | C{6cm} |}
  \hline
  Параметр & Субпараметри & Відповідний елемент у інтерфейсі \\
  \hline
  -dm
  & --
  & Режим :: Звичайний K-N-Back \\
  \hline
  -sm
  & --
  & Режим :: K-N-Back з обертами і здвигами за модулем \\
  \hline
  -n=<N>
  & N -- кількість ознак для запам'ятання
  & Довжина буфера :: N \\
  \hline
  -z
  & --
  & Додаткові просторові перетворення :: Z-оберти лицевою гранню \\
  \hline
  -c
  & --
  & Додаткові ознаки :: Колір \\
  \hline
  -s
  & --
  & Додаткові ознаки :: Звук \\
  \hline
  -bg
  & --
  & Графічні опції :: Космічний фон \\
  \hline
  -ic=<C>
  & C - колір
  & Графічні опції :: Колір спокою \\
  \hline
  -bc=<C>
  & C - колір
  & Графічні опції :: Колір активації \\
  \hline
  -bt=<T>
  & T - час (секунди)
  & Таймінги :: Час активації \\
  \hline
  -it=<T>
  & Т - час (секунди)
  & Таймінги :: Інтервали \\
  \hline
\end{longtable}\normalsize
\newpage
\subsection{Випробування програмного продукту}