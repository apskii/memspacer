\section{ТЕХНОЛОГІЧНИЙ РОЗДІЛ}
\subsection{Керівництво користувача}
Є два способи роботи із системою:
\begin{itemize}
  \item через інтерфейс користувача;
  \item безпосередньо через виконуваний файл тренажеру.
\end{itemize}

Перший спосіб є стандартним і рекомендованим усім звичайним користувачам, другий є досить низкорівневим, і призначеним в першу чергу для розробників альтернативних користувацьких інтерфейсів для тренажеру.
\subsubsection{Робота із системою через інтерфейс користувача}
Робота із системою цим шляхом передбачає запуск виконуваного файлу ``memspacer-ui'' (або ``memspacer-ui.exe'' на системах \emph{Windows}). Після цього користувач потрапляє до інтерфейсу, наведеного на рисунку ~\ref{fig:memspacer-ui}.
\begin{figure}[here]
  \centering\includegraphics[scale=0.6]{./memspacer-ui.png}
  \caption{Інтерфейс користувача}
  \label{fig:memspacer-ui}
\end{figure}

Вікно інтерфейсу має дві вкладки, назви яких відповідають їх призначенню:
\begin{itemize}
  \item ``Налаштування'';
  \item ``Істория тренувань''.
\end{itemize}

Випадаючий список над вкладками призначений для вибору профілю, а кнопки після нього, зліва-направо, для досягнення таких цілей:
\begin{itemize}
  \item створення нового профілю;
  \item збереження конфігурації поточного профілю;
  \item відміна змін до конфігурації поточного профілю.
\end{itemize}
\newpage
Призначення інших елементів інтерфейсу наведено у таблиці \ref{table:memspacer-config}.
\small\begin{longtable}{| C{6cm} | C{3cm} | C{6cm} |}
  \caption{Призначення елементів інтерфейсу користувача}
  \label{table:memspacer-config} \\
  \hline
  Елемент & Тип & Призначення \\
  \hline
  \endfirsthead
  \multicolumn{3}{r}%
    {{\normalsize Продовження таблиці \thetable\ }} \\
    \hline
  Елемент & Тип & Призначення \\
  \hline
  \endhead
  Режим :: Звичайний K-N-Back
  & Альтернатива (``режим'')
  & Використовувати під час тренування звичайний K-N-Back на одній грані \\
  \hline
  Режим :: K-N-Back з обертами і здвигами за модулем
  & Альтернатива (``режим'')
  & Використовувати під час тренування просторовий K-N-Back
  з обертами і здвигами за модулем \\
  \hline
  Довжина буфера :: N
  & Число
  & Задати кількість необхідних для запамятовування
  елементів буферу кожної із ознак \\
  \hline
  Додаткові просторові перетворення :: Z-оберти лицевою гранню
  & Флаг
  & Використовувати під час тренування повороти кубу навколо осі,
  перпендикулярної площині екрану \\
  \hline
  Додаткові ознаки :: Колір
  & Флаг
  & Використовувати під час тренування блимання клітин різними кольорами
  у якості додактової ознаки до запам'ятання \\
  \hline
  Додаткові ознаки :: Звук
  & Флаг
  & Використовувати під час тренування оголошення літер
  у якості додактової ознаки до запам'ятання \\
  \hline
  Графічні опції :: Космічний фон
  & Флаг
  & Використовувати під час тренування шейдер космосу для фону \\
  \hline
  Графічні опції :: Колір спокою
  & Колір
  & Колір, у якому перебуває клітина у стані спокою (не блимає)\\
  \hline
  Графічні опції :: Колір активації
  & Колір
  & Колір, якого набуває клітина під час блимання \\
  \hline
  Таймінги :: Час активації
  & Число (секунди)
  & Час, за який клітина набуває кольору блимання під час блимання \\
  \hline
  Таймінги :: Інтервали
  & Число (секунди)
  & Інтервали між демонстраціями рядів ознак \\
  \hline
\end{longtable}\normalsize
\subsubsection{Робота із системою безпосередньо через винонуваний файл програми моделювання}
Робота із системою цим шляхом передбачає запуск виконуваного файлу ``memspacer'' (або ``memspacer.exe'' на системах \emph{Windows}). Конфігурування здійснюється через аргументи командної строки, результати тренування виводяться на стандартний пристрій виводу після закінчення тренування.
\newpage
Опис параметрів командної строки наведено у таблиці \ref{table:memspacer-cmd}.
\small\begin{longtable}{| C{4cm} | C{5cm} | C{6cm} |}
  \caption{Параметри командної строки для безпосередньої роботи через виконуваний файл програми моделювання}
  \label{table:memspacer-cmd} \\
  \hline
  Параметр & Субпараметри & Відповідний елемент у інтерфейсі \\
  \hline
  \endhead
  -dm
  & --
  & Режим :: Звичайний K-N-Back \\
  \hline
  -sm
  & --
  & Режим :: K-N-Back з обертами і здвигами за модулем \\
  \hline
  -n=<N>
  & N -- кількість ознак для запам'ятання
  & Довжина буфера :: N \\
  \hline
  -z
  & --
  & Додаткові просторові перетворення :: Z-оберти лицевою гранню \\
  \hline
  -c
  & --
  & Додаткові ознаки :: Колір \\
  \hline
  -s
  & --
  & Додаткові ознаки :: Звук \\
  \hline
  -bg
  & --
  & Графічні опції :: Космічний фон \\
  \hline
  -ic=<C>
  & C - колір
  & Графічні опції :: Колір спокою \\
  \hline
  -bc=<C>
  & C - колір
  & Графічні опції :: Колір активації \\
  \hline
  -bt=<T>
  & T - час (секунди)
  & Таймінги :: Час активації \\
  \hline
  -it=<T>
  & Т - час (секунди)
  & Таймінги :: Інтервали \\
  \hline
\end{longtable}\normalsize

\subsubsection{Процес тренування}
Після запуску програми моделювання починається процес тренування. Незалежно від налаштувань, у ряді ознак, що надаються користувачу на кожному кроці, присутня ознака позиції клітини, що відображається блиманням останньої (рисунок \ref{fig:blink}). У режимі звичайного N-Back, користувач повинен сигналізувати співпадіння обраної системою клітини поточного кроку з клітиною N кроків назад. Якщо ж обрано N-Back з обертами і здвигами за модулем, між кожними двома блиманнями куб обертається на 90 градусів вліво, вправо, вгору, вниз, або перпендикулярно до камери, і користувач повинен здвигати запам'ятовані позиції за модулем у відповідному напрямку. Наприклад, при обертанні вправо (рисунок \ref{fig:rotation}), позиції потрібно відобразити у відповідності до схеми на рисунку \ref{fig:shift}.

В залежності від налаштувань, можуть бути присутні ознаки звуку та кольору. Перша полягає у тому, що користувачу на кожному кроці оголошується літера, і він повинен сигналізувати співпадіння поточної літери із літерою N кроків назад, а друга -- у тому, що блимання клітини відбувається кожен раз новим кольором і користувач повинен сигналізувати за тим самим принципом співпадіння кольорів.

Для сигналізування співпадіння ознак користувач повинен натискати кнопки на клавіатурі:
\begin{itemize}
  \item співпадіння позиції -- z;
  \item співпадіння літери -- x;
  \item співпадіння кольору -- c.
\end{itemize}

\begin{figure}[here]
  \centering\includegraphics[scale=0.25]{./diagrams/blink.png}
  \caption{Блимання клітини}
  \label{fig:blink}
\end{figure}

\begin{figure}[here]
  \centering\includegraphics[scale=0.25]{./diagrams/rotation.png}
  \caption{Обертання вправо}
  \label{fig:rotation}
\end{figure}

\begin{figure}[here]
  \centering\includegraphics[scale=0.6]{./diagrams/shift.eps}
  \caption{Відображення позицій при обертанні вправо}
  \label{fig:shift}
\end{figure}

\newpage
~\newpage
\subsection{Випробування програмного продукту}

У процесі тестування була перевірена функціональність системи. В таблицях \ref{table:create-profile}-\ref{table:answer-classification} наведений перелік випробувань основних функціональних можливостей програми конфігурування і перегляду історії та програми моделювання.

\small\begin{longtable}{| C{6cm} | C{10cm} |}
  \caption{Створення профілю}
  \label{table:create-profile} \\
  \hline
  Мета тесту & Перевірка функції ``Створення профілю'' \\
  \hline
  Початковий стан
  & Відкрита програма конфігурування і перегляду історії \\
  \hline
  Вхідні дані
  & Ім'я профілю \\
  \hline
  Схема проведення тесту
  & Натиснути кнопку ``Створити профіль''; увести ім'я нового профілю і натиснути ``Ок'' \\
  \hline
  Очікуваний результат
  & У списку профілів з'явився новий профіль із заданим ім'ям \\
  \hline
  Стан після проведення випробування
  & Відкрита програма конфігурування і перегляду історії із присутнім
  у списку профілів новим профілем \\
  \hline
\end{longtable}\normalsize

\small\begin{longtable}{| C{6cm} | C{10cm} |}
  \caption{Видалення профілю}
  \label{table:delete-profile} \\
  \hline
  Мета тесту & Перевірка функції ``Видалення профілю'' \\
  \hline
  Початковий стан
  & Відкрита програма конфігурування і перегляду історії;
  у списку профілів присутній той, що буде видалятися \\
  \hline
  Вхідні дані
  & -- \\
  \hline
  Схема проведення тесту
  & Вибрати профіль, що буде видалятися, із списку профілів;
  натиснути кнопку ``Видалити профіль'' \\
  \hline
  Очікуваний результат
  & Із списку профілів буде видалено заданий профіль,
  а поточний зміниться на будь-який інший \\
  \hline
  Стан після проведення випробування
  & Відкрита програма конфігурування і перегляду історії із
  відсутнім у списку профілів видаленим профілем \\
  \hline
\end{longtable}\normalsize
\newpage
\small\begin{longtable}{| C{6cm} | C{10cm} |}
  \caption{Збереження змін профілю}
  \label{table:save-profile-changes} \\
  \hline
  Мета тесту & Перевірка функції ``Збереження змін профілю'' \\
  \hline
  Початковий стан
  & Відкрита програма конфігурування і перегляду історії \\
  \hline
  Вхідні дані
  & -- \\
  \hline
  Схема проведення тесту
  & Змінити будь-які налаштування і натиснути кнопку ``Зберегти зміни профілю'';
  потім натиснути кнопку ``Відмінити зміни профілю'' \\
  \hline
  Очікуваний результат
  & Нові налаштування збережені для поточного профілю після натиснення першої кнопки,
  і тому натиснення другої не має ефекту \\
  \hline
  Стан після проведення випробування
  & Відкрита програма конфігурування і перегляду історії
  із збереженими змінами до поточного профілю \\
  \hline
\end{longtable}\normalsize

\small\begin{longtable}{| C{6cm} | C{10cm} |}
  \caption{Відміна змін профілю}
  \label{table:undo-profile-changes} \\
  \hline
  Мета тесту & Перевірка функції ``Відміна змін профілю'' \\
  \hline
  Початковий стан
  & Відкрита програма конфігурування і перегляду історії \\
  \hline
  Вхідні дані
  & -- \\
  \hline
  Схема проведення тесту
  & Змінити будь-які налаштування і натиснути кнопку ``Відмінити зміни профілю'' \\
  \hline
  Очікуваний результат
  & Налаштування повертаються у стан, в якому вони були до змін під час проведення тесту \\
  \hline
  Стан після проведення випробування
  & Відкрита програма конфігурування і перегляду історії \\
  \hline
\end{longtable}\normalsize

\small\begin{longtable}{| C{6cm} | C{10cm} |}
  \caption{Чистота історії створеного профілю}
  \label{table:new-profile-clean-history} \\
  \hline
  Мета тесту & Перевірка чистоти історії створеного профілю \\
  \hline
  Початковий стан
  & Відкрита програма конфігурування і перегляду історії \\
  \hline
  Вхідні дані
  & -- \\
  \hline
  Схема проведення тесту
  & Створити і вибрати новий профіль; перейти на вкладку ``Історія тренувань'' \\
  \hline
  Очікуваний результат
  & Історія тренувань повинна бути пустою \\
  \hline
  Стан після проведення випробування
  & Відкрита програма конфігурування і перегляду історії \\
  \hline
\end{longtable}\normalsize
\newpage
\small\begin{longtable}{| C{6cm} | C{10cm} |}
  \caption{Працездатність налаштувань}
  \label{table:config-correctness} \\
  \hline
  Мета тесту & Перевірка працездатності налаштувань \\
  \hline
  Початковий стан
  & Відкрита програма конфігурування і перегляду історії \\
  \hline
  Вхідні дані
  & -- \\
  \hline
  Схема проведення тесту
  & Перейти на вкладку ``Налаштування''; сконфігурувати деякі налаштування; почати сеанс тренування \\
  \hline
  Очікуваний результат
  & Під час тренування повинні бути присутні аспекти, відповідні заданим налаштуванням \\
  \hline
  Стан після проведення випробування
  & Відкрита програма моделювання або програма конфігурування та перегляду історії \\
  \hline
\end{longtable}\normalsize

\small\begin{longtable}{| C{6cm} | C{10cm} |}
  \caption{Коректність класифікації відповедей}
  \label{table:answer-classification} \\
  \hline
  Мета тесту & Перевірка коректності класифікації відповідей \\
  \hline
  Початковий стан
  & Відкрита програма конфігурування і перегляду історії \\
  \hline
  Вхідні дані
  & -- \\
  \hline
  Схема проведення тесту
  & Перейти на вкладку ``Історія тренувань''; запам'ятати наведені дані;
  провести сеанс тренування, записуючи його на відео;
  перейти на вкладку ``Історія тренувань'' \\
  \hline
  Очікуваний результат
  & В історії тренувань повинен був з'явитися новий запис із коректними показниками
  щодо кількості вірних, невірних, та пропущених відповідей, що можна перевірити
  за допомогою співставлення показників із відеозаписом \\
  \hline
  Стан після проведення випробування
  & Відкрита програма конфігурування і перегляду історії \\
  \hline
\end{longtable}\normalsize

\textbf{Висновок до розділу.} У даному розділі було наведено керівництво користувача для двох способів роботи із системою: через інтерфейс користувача і безпосередньо через виконуваний файл програми моделювання, а також були описані випробування системи.