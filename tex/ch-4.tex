\section{ПРОГРАМНЕ ТА ТЕХНІЧНЕ ЗАБЕЗПЕЧЕННЯ}
\subsection{Засоби розробки}
\subsubsection{Засоби розробки програми моделювання}

Для розробки програми моделювання були використані наступні засоби:

% [label=\Large$\diamond$]
\begin{itemize}\itemsep1em
  \item мова програмування -- C\verb!++!\cite{cpp} :

    Поєднує у собі можливості контролю над моделлю пам'яті із такими засобами абстрагування, як об'єктно-орієнтоване та кероване типами шаблонне узагальнене та метапрограмування, і, як наслідок, надає можливість реалізації спеціалізованих під потреби системи виразних засобів, і схем ефективного менеджменту пам'яті (наприклад, використання пулів памяті для частостворюваних короткоживучих об'єктів для запобігнення фрагментації пам'яті та зменшення затрат на алокації, тощо).

  \item середовище розробки -- KDevelop\cite{kdevelop}:

    Вільне середовище розробки програмного забезпечення для Linux, Solaris, FreeBSD, Mac OS X, Windows і різних Unix-систем, яке засноване на бібліотеках KDE/Qt і повністю підтримує процес розробки для KDE. KDevelop не включає в свій склад компілятор; натомість він використовує GNU Compiler Collection (або будь-який інший компілятор) для створення виконуваного коду. Первинною мовою розробки є C\verb!++!, але через використання плаґінів забезпечується підтримка додаткових мов програмування.

  \item графічна бібліотека -- OpenGL\cite{opengl}:

    Специфікація та бібліотека, що визначає незалежний від мови програмування кросплатформовий програмний інтерфейс (API) для написання застосунків, що використовують 2D та 3D комп'ютерну графіку. Цей інтерфейс містить понад 250 функцій, які можуть використовуватися для малювання складних тривимірних сцен з простих примітивів.

  \item інтеграційна бібліотека -- GLFW\cite{glfw}:

    Кросплатформова бібліотека для створення і відкриття вікон, створення OpenGL-контексту, управління вводом (клавіатурою, мишкою, джойстиком), часом, кліпбоардом, тощо.

  \item математична бібліотека -- GLM\cite{glm}:

    Header-only математична бібліотека для розробки графічних програм, заснованих на специфікації OpenGL Shading Language (GLSL). Надає класи та функції згідно із конвенціями GLSL. Окрім того, надає засоби роботи із матричними трансформаціями, кватерніонами, половинними чисельними типами, генерацією випадкових чисел, процедурним шумом, тощо.

\end{itemize}

\subsubsection{Засоби розробки програми конфігурування та перегляду історії}

Для розробки програми конфігурування та перегляду історії були використані наступні засоби:

\begin{itemize}\itemsep1em

  \item мова програмування -- Haskell\cite{haskell}:

    Функціональна мова програмування загального призначення із кешованим нормальним порядком обчислень (що також називають \emph{лінивими обчисленнями}) і найрозвиненішою серед прикладних мов сьогодення системою типів. Компілюється у нативний код. Обрана в першу чергу заради зручності, і в другу -- заради відсутності необхідності встановлення мовних рантаймів, таких як .NET Framework або JRE.

  \item середовище розробки -- GNU/Emacs\cite{emacs}:

    Потужний розширюваний, з великими можливостями у налаштуванні, екранний текстовий редактор. Є варіантом реалізації Emacs організації GNU. Створений Річардом Столменом. Вважається, що GNU Emacs доступний на найбільшій кількості аппаратних платформ серед усіх нетривіальних програмних систем. Може працювати як в текстовому режимі на текстових терміналах, так і в графічному в графічних середовищах.

  \item графічний інтерфейс -- GTK+\cite{gtk}:

    Кросплатформовий набір інструментів для створення графічних інтерфейсів користувача. До його складу входить повний набір віджетів, що дозволяють використовувати GTK+ для проектів різного рівня і розміру. GTK+ спеціально спроектований для підтримки не тільки C/C++, але й інших мов програмування, що в поєднанні з використанням візуального будівника інтерфейсу Glade дозволяє істотно спростити розробку і скоротити час написання графічних інтерфейсів.

  \item засіб побудови графічного інтерфейсу -- Glade\cite{glade}:

    Середовище для візуального проектування інтерфейсу на базі GTK+. Створений у Glade інтерфейс зберігається у форматі XML, який можна потім динамічно завантажити в GTK-застосунок за допомогою об'єкта GtkBuilder. XML-файли з визначенням інтерфейсу можуть бути використані в GTK-програмах на різних мовах програмування.

  \item СКБД -- SQLite\cite{sqlite}:

    Полегшена реляційна система керування базами даних, яка міститься у вигляді бібліотеки. Особливістю SQLite є те, що воно не використовує парадигму клієнт-сервер, тобто рушій SQLite не є окремим процесом, з яким взаємодіє застосунок, а надає бібліотеку, з якою програма компілюється і рушій стає складовою частиною програми. Таким чином, як протокол обміну використовуються виклики функцій (API) бібліотеки SQLite. Такий підхід зменшує накладні витрати, час відгуку і спрощує програму. SQLite зберігає всю базу даних (включаючи визначення, таблиці, індекси і дані) в єдиному стандартному файлі на тому комп'ютері, на якому виконується застосунок. Простота реалізації досягається за рахунок того, що перед початком виконання транзакції весь файл, що зберігає базу даних, блокується; ACID-функції досягаються зокрема за рахунок створення файлу-журналу.

\end{itemize}

\begin{comment}
\subsubsection{Засоби оформлення записки до проекту}
\begin{itemize}\itemsep1em
  \item Мова верстки -- TeX/LaTeX:

    Мова розмітки даних та пакет макросів TeX для високоякісного оформлення документів. Вважається стандартом де-факто для підготовки математичних і технічних текстів для публікації в наукових виданнях.
    На відміну від текстових процесорів, особливу увагу в LaTeX приділено відокремленню змісту статті від оформлення. LaTeX пропонує засоби для підготовки структурованих документів — документів, автор яких має можливість основну свою увагу зосередити на змісті, а оформлення і решту рутинної роботи перекласти на програму. Як і у випадку TeX — вхідні файли LaTeX можна порівняти із програмами.

  \item Движок рендерінгу документу -- XeTeX:

    Движок рендерінгу TeX з використанням Юнікоду та підтримкою сучасних шрифтових технологій, таких як OpenType, Graphite, та Apple Advanced Typography (AAT).
\end{itemize}
\end{comment}

\subsection{Вимоги до технічного забезпечення}

Для правильної роботи даної програми до складу технічних засобів повинні входити:

\begin{enumerate}
  \item Персональний комп'ютер з операційною системою Linux або Windows, задовольняючий наступним характеристикам:
  \begin{enumerate}
    \item процесор з тактовою частотою не нижче 400 Ггц;
    \item достатній об'єм оперативної памяті (не менше 64 МБ);
    \item наявність графічного процесору, задовольняючого наступним вимогам:
    \begin{itemize}
      \item об'єм відеопам'яті не менше ніж 32 МБ;
      \item підтримка OpenGL ES як мінімум версії 4.0;
    \end{itemize}
  \end{enumerate}
  \item Додатково на комп'ютері чи пристрої повинна бути доступною динамічна бібліотека трьовимірної графіки OpenGL;
  \item Периферія, до складу якої входить:
  \begin{enumerate}
    \item дисплей;
    \item кнопки (клавітура, тачскрин, чи кнопки пристрою);
    \item пристрій для відтворення звуку;
  \end{enumerate}
\end{enumerate}

\subsection{Архітектура програмного забезпечення}

\subsubsection{Ієрархія класів}

Ігровими об'єктами у системі ``Memspacer'' є дошка (Cube), її клітини (Cell), та сцена (StarNest). Вони відображені у ієрархії на рисунку \ref{fig:cd-gameobjects}.

\begin{figure}[here]
  \centering\includegraphics[scale=0.5]{./diagrams/cd-gameobjects.eps}
  \caption{Ієрархія ігрових об'єктів}
  \label{fig:cd-gameobjects}
\end{figure}

Ефектом є довільний вплив на стан об'єкту, який відбувається впродовж деякого проміжку часу. Коренем іерархії ефектів (рисунок \ref{fig:cd-effects}) є Effect<Target>, де Target — це тип об'єкту, на стан якого відбувається вплив.
Для комбінування ефектів відведені спеціальні класи:
Parallel<Target> комбінує два ефекти у новий ефект, в якому впливи двух довільних ефектів над Target відбуваються параллельно;
Sequential<Target> комбінує два ефекти у новий ланцюжковий ефект, в якому дія другого починається після закінчення проміжку впливу першого.
	Для зручної композиції ефектів вищеописаними комбінаторами реалізовано невеличкий eDSL (Embedded Domain Specific Language), відображений у ієрархії EffectTerm (рисунок \ref{fig:cd-effect-algebra}).

\begin{figure}[here]
  \centering\includegraphics[scale=0.4]{./diagrams/cd-effects.eps}
  \caption{Ієрархія ігрових ефектів}
  \label{fig:cd-effects}
\end{figure}

\begin{figure}[here]
  \centering\includegraphics[scale=0.4]{./diagrams/cd-effect-algebra.eps}
  \caption{Ієрархія термів ефектів}
  \label{fig:cd-effect-algebra}
\end{figure}

Для спрощення менеджменту шейдерів, вони структуровані ієрархією класів на рисунку \ref{fig:cd-shaders}.

\begin{figure}[here]
  \centering\includegraphics[scale=0.4]{./diagrams/cd-shaders.eps}
  \caption{Ієрархія шейдерів}
  \label{fig:cd-shaders}
\end{figure}

Ініціалізація графічного контексту, вікон, обробка вводу-виводу, та головний цикл гри відбуваються у класі Game. Глобальні параметри рендерингу, які є основним контекстом рендерингу, захоплені у класі RenderContext (рисунок \ref{fig:cd-misc}).

\begin{figure}[here]
  \centering\includegraphics[scale=0.4]{./diagrams/cd-misc.eps}
  \caption{Ієрархія загальних інтеграційних класів}
  \label{fig:cd-misc}
\end{figure}

~
\newpage
\subsubsection{Компоненти системи}

На рисунку \ref{fig:component} наведено структурну схему компонентів системи.

\begin{figure}[here]
  \centering\includegraphics[scale=0.6]{./diagrams/component.eps}
  \caption{Схема структурна компонентів}
  \label{fig:component}
\end{figure}

\newpage
\subsection{Специфікація функцій}

\subsubsection{Функції програми моделювання}

У таблиці \ref{table:funspec-a} наведено специфікацію функцій програми моделювання.

\small\begin{longtable}{| C{6cm} | C{4cm} | C{5cm} |}
  \caption{Функції програми моделювання}
  \label{table:funspec-a} \\
  \hline
  Ідентифікатор & Тип & Опис \\
  \hline
  \endfirsthead
  \multicolumn{3}{r}%
    {{\normalsize Продовження таблиці \thetable\ }} \\
    \hline
  Ідентифікатор & Тип & Опис \\
  \hline
  \endhead
  ``core/configuration.hpp'' \newline core:: \newline parse_configuration \newline (args)
  & vector<string> $\to$ Configuration
  & Формує об'єкт конфігурації із вектору параметрів командної строки \emph{args} \\
  \hline
  ``core/effect.hpp'' \newline core:: \newline Effect<T>::free \newline (pool)
  & Pool\& $\to$ void
  & Звільняє пам'ять, відведену під даний ефект у даному пулі \emph{pool} \\
  \hline
  ``core/effect.hpp'' \newline core:: \newline Effect<T>::process \newline (target, delta)
  & (T\&, float) $\to$ bool
  & Застосовує дію даного ефекту на проміжку часу \emph{delta}
  до цілі \emph{target} та повертає логічне значення, що показує,
  чи завершився час дії ефекту після даної ітерації моделювання \\
  \hline
  ``core/render_context.hpp'' \newline core:: \newline
  RenderContext::new \newline (view, proj)
  & (const Mat4\&, const Mat4\&) $\to$ RenderContext
  & Конструює контекст рендерінгу з заданими початковими
  матрицями виду \emph{view} і проекції \emph{proj} \\

  %%%%%% core/game_object
  \hline
  ``core/game_object.hpp'' \newline core:: \newline
  GameObject::update \newline (delta, pool)
  & (float, Pool\&) $\to$ void
  & Моделює розвиток стану об'єкту на проміжку часу
  \emph{delta} із пулом ефектів \emph{pool} \\
  \hline
  ``core/game_object.hpp'' \newline core:: \newline
  GameObject::render \newline (render_ctx)
  & RenderContext\& $\to$ void
  & Виконує дії з графічного відображення даного об'єкту на екран
  за заданого контексту рендерінгу \emph{render_ctx} \\
  \hline
  ``core/game_object.hpp'' \newline core:: \newline
  GameObjectTemplate<Self>:: \newline update \newline (delta, pool)
  & (float, Pool\&) $\to$ void
  & Стандартна реалізація \emph{GameObject::render},
  що поновлює із заданими параметрами усі ефекти,
  застосовані до даного ігрового об'єкту \\
  \hline
  ``core/game_object.hpp'' \newline core:: \newline
  GameObjectTemplate<Self>:: \newline attach_effect \newline (effect)
  & Effect<Self>\& $\to$ void
  & Застосовує заданий ефект до даного ігрового об'єкту,
  та вносить його у список ефектів цього об'єкту \\

  %%%%%% core/shader
  \hline
  ``core/shader.hpp'' \newline core:: \newline
  Shader::load \newline (shader_type, file_path)
  & (GLenum, string) $\to$ GLuint
  & Завантажує шейдер заданого типу \emph{shader_type} із файлу,
  розташованого за шляхом \emph{file_path}, і повертає його ідентифікатор \\
  \hline
  ``core/shader.hpp'' \newline core:: \newline
  Shader::use
  & void
  & Встановлює даний шейдер частиною поточного стану рендерінгу \\
  \hline
  ``core/shader.hpp'' \newline core:: \newline
  Shader::new \newline (descriptions)
  & initializer_list \newline <(GLenum,string)> \newline $\to$ Shader
  & Конструює шейдерну програму із набору пар-описів, кожна з яких
  складається з типу шейдеру та шляху до його файлу \\

  %%%%%% effect/parallel
  \hline
  ``effect/parallel.hpp'' \newline effect:: \newline
  Parallel<T>::new \newline (first, second)
  & (Effect<T>\&, Effect<T>\&) $\to$ Effect
  & Паралельна композиція ефектів \emph{first} та \emph{second} \\
  \hline
  ``effect/parallel.hpp'' \newline effect:: \newline
  Parallel<T>::free \newline (pool)
  & Pool\& $\to$ void
  & Звільняє пам'ять, відведену під даний ефект, та його складові ефекти, у даному пулі \\
  \hline
  ``effect/parallel.hpp'' \newline effect:: \newline Parallel<T>::process \newline (target, delta)
  & (T\&, float) $\to$ bool
  & Застосовує дію даної композиції ефектів на проміжку часу \emph{delta}
  до цілі \emph{target} та повертає логічне значення, що показує,
  чи завершився час дії ефектів після даної ітерації моделювання \\

  %%%%%% effect/sequential
  \hline
  ``effect/sequential.hpp'' \newline effect:: \newline
  Sequential<T>::new \newline (first, second)
  & (Effect<T>\&, Effect<T>\&) $\to$ Effect
  & Послідовна композиція ефектів \emph{first} та \emph{second} \\
  \hline
  ``effect/sequential.hpp'' \newline effect:: \newline
  Sequential<T>::free \newline (pool)
  & Pool\& $\to$ void
  & Звільняє пам'ять, відведену під даний ефект, та його складові ефекти, у даному пулі \\
  \hline
  ``effect/sequential.hpp'' \newline effect:: \newline Sequential<T>::process \newline (target, delta)
  & (T\&, float) $\to$ bool
  & Застосовує дію даної композиції ефектів на проміжку часу \emph{delta}
  до цілі \emph{target} та повертає логічне значення, що показує,
  чи завершився час дії ефектів після даної ітерації моделювання \\

  %%%%%% effect/algebra
  \hline
  ``effect/algebra.hpp'' \newline effect:: \newline EffectTerm<Self>:: \newline
  (||)<ET> \newline (other)
  & ET $\to$ \newline ParEffectTerm \newline <Self, ET>
  & Комбінатор термів паралельної композиції ефектів \\
  \hline
  ``effect/algebra.hpp'' \newline effect:: \newline EffectTerm<Self>:: \newline
  (>>)<ET> \newline (other)
  & ET $\to$ \newline SeqEffectTerm \newline <Self, ET>
  & Комбінатор термів послідовної композиції ефектів \\
  \hline
  ``effect/algebra.hpp'' \newline effect:: \newline WrapEffectTerm<T>:: \newline
  eval \newline (pool)
  & Pool\& $\to$ Effect<T>*
  & Інтерпретує врапаючі терми підмови опису ефектів з заданим пулом \\
  \hline
  ``effect/algebra.hpp'' \newline effect:: \newline ParEffectTerm<ET1,ET2>:: \newline
  eval \newline (pool)
  & Pool\& $\to$ Effect<T>*
  & Інтерпретує паралельно-композиційні терми підмови опису ефектів з заданим пулом \\
  \hline
  ``effect/algebra.hpp'' \newline effect:: \newline SeqEffectTerm<ET1,ET2>:: \newline
  eval \newline (pool)
  & Pool\& $\to$ Effect<T>*
  & Інтерпретує послідовно-композиційні терми підмови опису ефектів з заданим пулом \\

  %%%%%% effects/rotation
  \hline
  ``effects/rotation.hpp'' \newline effects:: \newline Rotation<Oriented>:: \newline
  new \newline (duration, orientation)
  & (float, Quat) $\to$ Rotation<Oriented>
  & Конструює ефект обертання з заданою тривалістю \emph{duration}
  та кватерніоном оберту \emph{orientation} \\
  \hline
  ``effects/rotation.hpp'' \newline effects:: \newline Rotation<Oriented>:: \newline
  process \newline (target, delta)
  & (Oriented\&, float) $\to$ bool
  & Застосовує описуєме даним об'єктом обертання на проміжку часу \emph{delta}
  до цілі \emph{target} та повертає логічне значення, що показує,
  чи завершився час дії ефекту після даної ітерації моделювання \\

  %%%%%% effects/blink
  \hline
  ``effects/blink.hpp'' \newline effects:: \newline Blink<Colored>:: \newline
  new \newline (duration, color)
  & (float, Vec4) $\to$ Rotation<Oriented>
  & Конструює ефект блимання з заданою тривалістю \emph{duration} та кольором \emph{color} \\
  \hline
  ``effects/blink.hpp'' \newline effects:: \newline Blink<Colored>:: \newline
  process \newline (target, delta)
  & (Colored\&, float) $\to$ bool
  & Застосовує описуєме даним об'єктом блимання на проміжку часу \emph{delta}
  до цілі \emph{target} та повертає логічне значення, що показує,
  чи завершився час дії ефекту після даної ітерації моделювання \\

  %%%%%% game_objects/cell
  \hline
  ``game_objects/cell.hpp'' \newline game_objects:: \newline Cell<Owner>:: \newline
  new \newline (parent, position, orientation, color)
  & (Owner*, Vec3, Quat, Vec4) $\to$ Cell<Owner>
  & Конструює клітину дошки із заданими дошкою \emph{parent},
  розташуванням \emph{position},  орієнтацією \emph{orientation},
  та кольором \emph{color} \\
  \hline
  ``game_objects/cell.hpp'' \newline game_objects:: \newline Cell<Owner>:: \newline
  render \newline (render_ctx)
  & const RenderContext\& $\to$ void
  & Виконує дії з графічного відображення клітини дошки на екран
  за заданого контексту рендерінгу \emph{render_ctx} \\

  %%%%%% game_objects/cube
  \hline
  ``game_objects/cube.hpp'' \newline game_objects:: \newline Cube:: \newline
  new \newline (position, orientation)
  & (Vec3, Quat) $\to$ Cube
  & Конструює дошку із заданими розташуванням \emph{position} і орієнтацією \emph{orientation} \\
  \hline
  ``game_objects/cube.hpp'' \newline game_objects:: \newline Cube:: \newline
  render \newline (render_ctx)
  & const RenderContext\& $\to$ void
  & Виконує дії з графічного відображення дошки на екран
  за заданого контексту рендерінгу \emph{render_ctx} \\
  \hline
  ``game_objects/cube.hpp'' \newline game_objects:: \newline Cube:: \newline
  update \newline (delta, pool)
  & (float, Pool\&) $\to$ void
  & Моделює розвиток стану дошки на проміжку часу \emph{delta} із пулом ефектів \emph{pool} \\

  %%%%%% game_objects/star_nest
  \hline
  ``game_objects/star_nest.hpp'' \newline game_objects:: \newline StarNest:: \newline
  new
  & StarNest
  & Конструює зоряне небо \\
  \hline
  ``game_objects/star_nest.hpp'' \newline game_objects:: \newline StarNest:: \newline
  render \newline (render_ctx)
  & const RenderContext\& $\to$ void
  & Виконує дії з графічного відображення зоряного неба на екран
  за заданого контексту рендерінгу \emph{render_ctx} \\

  %%%%%% shaders/...
  \hline
  ``shaders/default.hpp'' \newline shaders:: \newline Default:: \newline
  new
  & shaders::Default
  & Конструює стандартний шейдер, що відповідає за освітлення
  і кольори клітин та їх MVP перетворення \\
  \hline
  ``shaders/star_nest.hpp'' \newline shaders:: \newline StarNest:: \newline
  new
  & shaders::StarNest
  & Конструює шейдер зоряного неба \\

  %%%%%% core/game
  \hline
  ``core/game.hpp'' \newline core:: \newline Game:: \newline
  init_ogl
  & GLFWWindow*
  & Ініціалізує OpenGL, GLFW, та GLEW \\
  \hline
  ``core/game.hpp'' \newline core:: \newline Game:: \newline
  new \newline (window, cfg)
  & (GLFWWindow*, Configuration) $\to$ Game
  & Конструює ігровий контекст із заданими вікном \emph{window}
  та конфігурацією \emph{cfg} \\
  \hline
  ``core/game.hpp'' \newline core:: \newline Game:: \newline
  die
  & void
  & Примусово завершує моделювання \\
  \hline
  ``core/game.hpp'' \newline core:: \newline Game:: \newline
  render
  & void
  & Виконує ітерацію рендерингу усіх об'єктів моделювання \\
  \hline
  ``core/game.hpp'' \newline core:: \newline Game:: \newline
  update \newline (delta)
  & float $\to$ void
  & Моделює розвиток стану усієї системи на проміжку часу \emph{delta} \\
  \hline
  ``core/game.hpp'' \newline core:: \newline Game:: \newline
  run \newline (cfg)
  & const Configuration\& $\to$ int
  & Запускає моделювання з заданою конфігурацією \\
  \hline
\end{longtable}\normalsize
\newpage
\subsubsection{Функції програми конфігурування та перегляду історії}

У таблиці \ref{table:funspec-b} наведено специфікацію функцій програми конфігурування та перегляду історії.

\small\begin{longtable}{| C{6cm} | C{4cm} | C{5cm} |}
  \caption{Функції програми конфігурування та перегляду історії}
  \label{table:funspec-b} \\
  \hline
  Ідентифікатор & Тип & Опис \\
  \hline
  \endfirsthead
  \multicolumn{3}{r}%
    {{\normalsize Продовження таблиці \thetable\ }} \\
    \hline
  Ідентифікатор & Тип & Опис \\
  \hline
  \endhead
  ``MemspacerUI.hs'' \newline (<:=) \newline ref val
  & IORef a $\to$ a $\to$ IO ()
  & Операторний синонім до writeIORef: записує значення \emph{val}
  у IO-змінну \emph{ref} \\
  \hline
  ``MemspacerUI.hs'' \newline (<:<) \newline ref mval
  & IORef a $\to$ IO a $\to$ IO ()
  & Комбінація монадичного зв'язування і (<:=):
  запускає монадичне обчислення \emph{mval} і записує
  у IO-змінну \emph{ref} його результат \\
  \hline
  ``MemspacerUI.hs'' \newline cmdArgs \newline cfg
  & Config $\to$ String
  & Перетворює об'єкт конфігурації у строку параметрів командної строки \\
  \hline
  ``MemspacerUI.hs'' \newline activeProfile
  & SqlPersistM String
  & Отримує з бази даних ім'я активного профілю \\
  \hline
  ``MemspacerUI.hs'' \newline configFor \newline profile
  & String $\to$ SqlPersistM (Key Config, Config)
  & Отримує з бази даних конфігурацію для профіля з ім'ям \emph{profile} \\
  \hline
  ``MemspacerUI.hs'' \newline activeConfig
  & SqlPersistM (Key Config, Config)
  & Отримує з бази даних активну конфігурацію \\
  \hline
  ``MemspacerUI.hs'' \newline db
  & SqlPersistM a $\to$ IO a
  & Запускає SqlPersist-монадичне обчислення у IO-монаді \\
  \hline
  ``MemspacerUI.hs'' \newline whenM \newline mb action
  & Monad m $\Rightarrow$ m Bool $\to$ m () $\to$ m ()
  & Запускає монадичне обчислення \emph{mb} і у разі істинності
  отриманого результату запускає монадичне обчислення \emph{action} \\
  \hline
  ``MemspacerUI.hs'' \newline toolButtonClicked
  & Signal ToolButton (IO ())
  & Сигнал, що виникає при натисненні на ToolButton \\
  \hline
  ``MemspacerUI.hs'' \newline ($\sim$\$) \newline widget signal action
  & obj \newline $\to$ Signal obj cb \newline
  $\to$ (obj $\to$ cb) \newline $\to$ IO (ConnectId obj)
  & Встановлює реакцію \emph{action} на сигнал \emph{signal} віджету \emph{widget} \\
  \hline
  ``MemspacerUI.hs'' \newline columnInserter \newline treeView model
  & TreeView \newline $\to$ ListStore s $\to$ ColumnInserter s
  & Створює фабрику колонок для даного дерева та моделі \\
  \hline
  ``MemspacerUI.hs'' \newline initHistory \newline historyTV
  & TreeView $\to$ IO (ListStore Session)
  & Ініціалізує дерево \emph{historyTV} для відображення історії тренувань,
  створює, прив'язує до дерева, і повертає необхідну модель даних \\
  \hline
  ``MemspacerUI.hs'' \newline addProfileDialog
  & IO (IO (Maybe String))
  & Конструює діалог створення профілю, що повертає і'мя нового профілю \\
  \hline
  ``MemspacerUI.hs'' \newline memspacerUI \newline
  initialConfig initialProfilesList
  & (Key Config, Config) $\to$ [String] $\to$ IO ()
  & Ініціалізує інтерфейс конфігурування і перегляду історії
  з заданими початковою конфігурацією та початковим списком профілів \\
  \hline
\end{longtable}\normalsize

\textbf{Висновок до розділу.} У даному розділі було розглянуто програмне і технічне забезпечення системи окремо для програми моделювання і програми конфігурування та перегляду історії; проаналізовано архітектуру системи; побудовані структурні схеми класів і компонентів; наведено специфікації функцій.