\section{Математичне забезпечення}

\subsection{Кватерніони}

Кватерніон  — гіперкомплексне число, яке реалізується в 4-вимірному просторі. Вперше описане В. Р. Гамільтоном у 1843 році.

Кватерніони використовуються як у теоретичній, так і у прикладній математиці, зокрема для розрахунку поворотів у просторі у тривимірній графіці та машинному зорі.

Кватерніон має вигляд $a + bi + cj + dk$, де $a, b, c, d$ — дійсні числа;
$i, j, k$ — уявні одиниці, що задовольняють співвідношенням
$i^2 = j^2 = k^2 = ijk = -1$, з яких випливають також наступні співвідношення: \\
\begin{center}
  $\begin{matrix}
    ij & = & -ji & = & k, \\
    jk & = & -kj & = & i, \\
    ki & = & -ik & = & j.
  \end{matrix}$
\end{center}

Часто замість $i, j, k$ використовують позначення для уявних одиниць відповідно $i_1, i_2, i_3$, а також покладають $i_0 := 1$.

Ще один, зрідка вживаний, варіант позначень: $e_0, e_1, e_2, e_3$.

Кватерніони також можна визначити через комплексні числа, використовуючи процедуру подвоєння Келі-Діксона.

\subsection{Сферична лінійна інтерполяція}

\subsubsection{Геометрична інтерпретація}
В комп'ютерній графіці, SLERP (spherical linear interpolation) — лінійна інтерполяція на сфері, що використовується для анімації обертання з постійною кутовою швидкістю за допомогою кватерніонів.

SLERP має геометричну інтерпретацію незалежну від кватерніонів та розмірності простору. Вона базується на тому, що довільна точка на кривій повинна представлятись у вигляді лінійної комбінації кінців кривої. Якщо простором, в якому беруться точки, буде сфера, то геодезичний відрізок, який їх з'єднує буде не евклідовим відрізком (він не належить сфері), а буде дугою великого кола на сфері.

Якщо p0 та p1 початок і кінець дуги, а t параметр, $0 \le t \le 1$.

Обчислимо $\Omega$ — кут дуги, отримаємо $\cos{\Omega} = p_0 \cdot p_1$, n-вимірний скалярний добуток одиничних векторів. Отримаємо формулу:
\begin{center}
  $\operatorname{Slerp}(p_0, \, p_1, \, t) = \frac{\sin {(1-t)\Omega}}{\sin \Omega} p_0 + \frac{\sin t\Omega}{\sin \Omega} p_1.$
\end{center}

Вона симетрична відносно кінців дуги
$\operatorname{Slerp}(p_1,p_1,t) = \operatorname{Slerp}(p_1,p_0,1-t)$.

\subsubsection{Запис за допомогою кватерніонів}

Записавши одиничний кватерніон у вигляді $q = \cos{\Omega} + v\sin{\Omega}$, де v - тривимірний одиничний вектор, отримаємо $qt = \cos{t\Omega} + v\sin{t\Omega}$.

Записавши $q = q_1q_0^{-1}$, отримаємо

$\mathrm{Slerp}(q_0, q_1, t) = q_0^{1-t} q_1^t
=q_0 (q_0^{-1} q_1)^t
=q_1 (q_1^{-1} q_0)^{1-t}
=(q_0 q_1^{-1})^{1-t} q_1
=(q_1 q_0^{-1})^t q_0
$