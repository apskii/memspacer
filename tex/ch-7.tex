\section*{ЗАГАЛЬНІ ВИСНОВКИ}
\addcontentsline{toc}{section}{ЗАГАЛЬНІ ВИСНОВКИ}%

У ході виконання дипломного проекту було детально розглянуто вправу для розвитку короткочасної пам'яті N-Back, сформульовані її модифікації та комплекс задач із її розширення.

Був проведений ґрунтовний аналіз предметного середовища, ретельно описані варіанти використань і функціональна модель.

Для виникших у процесі розробки графічних задач, були сформульовані відповідні математичні задачі, для вирішення яких був підібраний, проаналізований, описаний, і застосований, математичний апарат матриць перетворень і кватерніонів.

Система була декомпозована на два застосування: програму моделювання вправи, і програму конфігурування та перегляду історії. Для розробки першої було використано мову С++, графічну бібліотеку OpenGL, математичну GLM, звукову SDL_mixer, для розробки другої -- мову Haskell, бібліотеку користувацьких інтерфейсів GTK+.

Було спроектовано базу даних, яка дає змогу ефективно та надійно здійснювати доступ до записів, що виникають після проведення користувачем серії тренувань. Для управління базою даних обрана СКБД SQLite.

Наведена детальна інструкція користувача з експлуатації комплексу задач, описана методика проведення випробувань, яка показує можливість введення програми в експлуатацію.
